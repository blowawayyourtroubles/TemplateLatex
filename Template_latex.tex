\documentclass[12pt,a4paper]{article}

% ========================================
% PAQUETES ESENCIALES
% ========================================
\usepackage[utf8]{inputenc}
\usepackage[spanish]{babel}
\usepackage{amsmath, amssymb, amsthm}
\usepackage{mathtools}
\usepackage{xcolor}
\usepackage{geometry}
\usepackage{graphicx}
\usepackage{booktabs}
\usepackage{array}
\usepackage{tabularx}
\usepackage{multicol}
\usepackage{enumitem}
\usepackage{fancyhdr}
\usepackage{titlesec}
\usepackage[most]{tcolorbox}
\usepackage{tikz}
\usepackage{pgfplots}
\usepackage{lmodern} % Fuente moderna
\usepackage{hyperref} % Enlaces clickeables en el PDF

\pgfplotsset{compat=1.18}
\geometry{margin=1in, top=1.2in, bottom=1in}

% ========================================
% CONFIGURACIÓN DE HYPERREF
% ========================================
\hypersetup{
    colorlinks=true,
    linkcolor=accent,
    filecolor=primary,
    urlcolor=primary,
    citecolor=secondary,
    pdftitle={Template Matemáticas Institucional},
    pdfauthor={Nombre del Autor},
    pdfsubject={Matemáticas},
    pdfkeywords={LaTeX, Matemáticas, Template},
    bookmarksnumbered=true,
    bookmarksopen=true
}

% ========================================
% PALETA DE COLORES INSTITUCIONAL
% ========================================
\definecolor{primary}{HTML}{00AF54}     
\definecolor{secondary}{HTML}{00AF54}    
\definecolor{accent}{HTML}{DEE0D8}       
\definecolor{highlight}{HTML}{EEF4D4}    
\definecolor{success}{HTML}{06a77d}      
\definecolor{warning}{HTML}{f77f00}      
\definecolor{danger}{HTML}{d62828}       
\definecolor{dark}{HTML}{272727}         
\definecolor{light}{HTML}{f8f9fa}        
\definecolor{background}{HTML}{4C5F6B} 

% ========================================
% CONFIGURACIÓN DE PÁGINA
% ========================================
\pagecolor{dark}
\color{white}

% ========================================
% ENCABEZADO Y PIE DE PÁGINA
% ========================================
\pagestyle{fancy}
\fancyhf{}
\fancyhead[L]{\textcolor{accent}{\textbf{Institución / Departamento}}}
\fancyhead[R]{\textcolor{highlight}{\nouppercase{\leftmark}}}
\fancyfoot[C]{\textcolor{accent}{\thepage}}
\renewcommand{\headrulewidth}{1.5pt}
\renewcommand{\footrulewidth}{1pt}
\renewcommand{\headrule}{\hbox to\headwidth{\color{primary}\leaders\hrule height \headrulewidth\hfill}}
\renewcommand{\footrule}{\hbox to\headwidth{\color{primary}\leaders\hrule height \footrulewidth\hfill}}

% ========================================
% FORMATO DE TÍTULOS Y SECCIONES
% ========================================
\titleformat{\section}
{\color{accent}\Large\bfseries}
{\textcolor{primary}{\thesection.}}{0.5em}{}
[\textcolor{primary}{\titlerule[1.5pt]}]

\titleformat{\subsection}
{\color{highlight}\large\bfseries}
{\textcolor{accent}{\thesubsection}}{0.5em}{}
[\textcolor{accent}{\titlerule[0.8pt]}]

\titleformat{\subsubsection}
{\color{white}\normalsize\bfseries}
{\textcolor{highlight}{\thesubsubsection}}{0.5em}{}

% ========================================
% CAJAS PERSONALIZADAS
% ========================================

% Caja de Definición
\newtcolorbox{definicion}[2][]{
	colback=dark,
	colframe=primary,
	coltext=white,
	boxrule=2pt,
	arc=4mm,
	leftrule=8mm,
	title={\textcolor{white}{DEFINICIÓN: #2}},
	fonttitle=\bfseries\large,
	coltitle=white,
	attach boxed title to top left={yshift=-3mm, xshift=5mm},
	boxed title style={colback=primary, arc=3mm},
	#1
}

% Caja de Teorema
\newtcolorbox{teorema}[2][]{
	colback=dark,
	colframe=secondary,
	coltext=white,
	boxrule=2pt,
	arc=4mm,
	leftrule=8mm,
	title={\textcolor{white}{TEOREMA: #2}},
	fonttitle=\bfseries\large,
	coltitle=white,
	attach boxed title to top left={yshift=-3mm, xshift=5mm},
	boxed title style={colback=secondary, arc=3mm},
	#1
}

% Caja de Ejemplo
\newtcolorbox{ejemplo}[2][]{
	colback=dark,
	colframe=success,
	coltext=white,
	boxrule=1.5pt,
	arc=3mm,
	title={\textcolor{white}{EJEMPLO: #2}},
	fonttitle=\bfseries,
	coltitle=white,
	attach boxed title to top left={yshift=-2mm, xshift=5mm},
	boxed title style={colback=success, arc=2mm},
	#1
}

% Caja de Nota/Observación
\newtcolorbox{nota}[2][]{
	colback=dark,
	colframe=warning,
	coltext=white,
	boxrule=1.5pt,
	arc=3mm,
	title={\textcolor{white}{NOTA: #2}},
	fonttitle=\bfseries,
	coltitle=white,
	attach boxed title to top left={yshift=-2mm, xshift=5mm},
	boxed title style={colback=warning, arc=2mm},
	#1
}

% Caja de Advertencia
\newtcolorbox{advertencia}[2][]{
	colback=dark,
	colframe=danger,
	coltext=white,
	boxrule=2pt,
	arc=3mm,
	title={\textcolor{white}{ADVERTENCIA: #2}},
	fonttitle=\bfseries,
	coltitle=white,
	attach boxed title to top left={yshift=-2mm, xshift=5mm},
	boxed title style={colback=danger, arc=2mm},
	#1
}

% Caja Simple para Fórmulas
\newtcolorbox{formulabox}[1][]{
	colback=background,
	colframe=accent,
	coltext=white,
	boxrule=1pt,
	arc=2mm,
	left=5pt,
	right=5pt,
	top=5pt,
	bottom=5pt,
	#1
}

% Caja de Ejercicio (NUEVA)
\newtcolorbox{ejercicio}[2][]{
	colback=dark,
	colframe=highlight,
	coltext=white,
	boxrule=1.5pt,
	arc=3mm,
	title={\textcolor{dark}{EJERCICIO: #2}},
	fonttitle=\bfseries,
	coltitle=dark,
	attach boxed title to top left={yshift=-2mm, xshift=5mm},
	boxed title style={colback=highlight, arc=2mm},
	#1
}

% ========================================
% ENTORNOS MATEMÁTICOS PERSONALIZADOS
% ========================================
\newtheoremstyle{customtheorem}
{10pt}{10pt}{\color{white}}{}{\color{accent}\bfseries}{:}{.5em}{}

\theoremstyle{customtheorem}
\newtheorem{thm}{Teorema}[section]
\newtheorem{lem}[thm]{Lema}
\newtheorem{prop}[thm]{Proposición}
\newtheorem{cor}[thm]{Corolario}
\newtheorem{defn}[thm]{Definición}

% ========================================
% COMANDOS PERSONALIZADOS
% ========================================

% Separadores visuales
\newcommand{\separator}{
	\vspace{10pt}
	\noindent\textcolor{primary}{\rule{\textwidth}{1.5pt}}
	\vspace{10pt}
}

\newcommand{\thinline}{
	\vspace{5pt}
	\noindent\textcolor{accent}{\rule{\textwidth}{0.5pt}}
	\vspace{5pt}
}

% Destacar texto
\newcommand{\highlight}[1]{\textcolor{accent}{\textbf{#1}}}
\newcommand{\important}[1]{\textcolor{warning}{\textbf{#1}}}
\newcommand{\critical}[1]{\textcolor{danger}{\textbf{#1}}}

% Fórmula destacada
\newcommand{\mainformula}[1]{
	\begin{center}
		\colorbox{secondary}{\textcolor{white}{$\displaystyle #1$}}
	\end{center}
}

% Conjuntos numéricos
\newcommand{\R}{\mathbb{R}}
\newcommand{\N}{\mathbb{N}}
\newcommand{\Z}{\mathbb{Z}}
\newcommand{\Q}{\mathbb{Q}}
\newcommand{\C}{\mathbb{C}}

% Vectores y matrices
\newcommand{\vect}[1]{\mathbf{#1}}
\newcommand{\abs}[1]{\left| #1 \right|}
\newcommand{\norm}[1]{\left\| #1 \right\|}

% Derivadas
\newcommand{\dv}[2]{\frac{d#1}{d#2}}
\newcommand{\dvn}[3]{\frac{d^{#3}#1}{d#2^{#3}}}
\newcommand{\pdv}[2]{\frac{\partial #1}{\partial #2}}

% Integrales
\newcommand{\inte}[2]{\int_{#1}^{#2}}
\newcommand{\intinf}{\int_{-\infty}^{\infty}}

% ========================================
% INFORMACIÓN DEL DOCUMENTO
% ========================================
\title{
	\textcolor{accent}{\Huge\textbf{Título del Documento}}\\
	\vspace{5pt}
	\textcolor{highlight}{\Large Subtítulo o Tema Específico}
}
\author{
	\textcolor{white}{\Large Nombre del Autor}\\
	\textcolor{accent}{Institución / Departamento de Matemáticas}\\
	\textcolor{highlight}{\texttt{correo@institucion.edu}}
}
\date{\textcolor{accent}{\today}}

% ========================================
% INICIO DEL DOCUMENTO
% ========================================
\begin{document}

% Portada
\maketitle
\thispagestyle{empty}

\vspace{20pt}
\begin{center}
\colorbox{primary}{\parbox{0.85\textwidth}{\color{white}\centering\large
Template profesional para documentos matemáticos con estilo institucional moderno. Incluye cajas personalizadas, comandos especializados, gráficas con TikZ/PGFplots, y formato completamente personalizable.
}}
\end{center}

\newpage
\tableofcontents
\newpage

% ========================================
% SECCIÓN 1: CAJAS Y ENTORNOS
% ========================================
\section{Cajas y Entornos Especiales}

\subsection{Definiciones y Teoremas}

\begin{definicion}{Límite de una función}
Sea $f: \R \to \R$ una función y $a \in \R$. Decimos que el \highlight{límite} de $f$ cuando $x$ tiende a $a$ es $L$ si:
\[
\forall \varepsilon > 0, \exists \delta > 0 : 0 < \abs{x - a} < \delta \implies \abs{f(x) - L} < \varepsilon
\]
Notación: $\displaystyle\lim_{x \to a} f(x) = L$
\end{definicion}

\vspace{10pt}

\begin{teorema}{Regla de la Cadena}
Sean $f: \R \to \R$ y $g: \R \to \R$ funciones derivables. Entonces la composición $f \circ g$ es derivable y:
\mainformula{(f \circ g)'(x) = f'(g(x)) \cdot g'(x)}
\end{teorema}

\vspace{10pt}

\begin{ejemplo}{Derivada de una función compuesta}
Sea $h(x) = \sin(x^2)$. Identificamos $f(u) = \sin u$ y $g(x) = x^2$.

Aplicando la regla de la cadena:
\begin{align*}
h'(x) &= f'(g(x)) \cdot g'(x) \\
&= \cos(x^2) \cdot 2x \\
&= 2x\cos(x^2)
\end{align*}
\end{ejemplo}

\vspace{10pt}

\begin{nota}{Continuidad y derivabilidad}
Si una función es derivable en un punto, entonces es continua en ese punto. Sin embargo, \important{el recíproco no es cierto}. Ejemplo: $f(x) = \abs{x}$ es continua en $x=0$ pero no derivable.
\end{nota}

\vspace{10pt}

\begin{advertencia}{Error común}
\critical{NO} confundir la derivada de un producto con el producto de derivadas:
\[
(f \cdot g)' \neq f' \cdot g'
\]
La regla correcta es: $(f \cdot g)' = f' \cdot g + f \cdot g'$ (Regla del producto)
\end{advertencia}

\vspace{10pt}

\begin{ejercicio}{Para practicar}
Calcular las siguientes derivadas:
\begin{enumerate}
\item $\dv{}{x}\left(e^{x^2}\right)$
\item $\dv{}{x}\left(\ln(\cos x)\right)$
\item $\dv{}{x}\left(x^x\right)$ \textit{(Sugerencia: usar logaritmos)}
\end{enumerate}
\end{ejercicio}

\separator

% ========================================
% SECCIÓN 2: TABLAS Y FÓRMULAS
% ========================================
\section{Tablas de Fórmulas}

\subsection{Derivadas de Funciones Básicas}

\begin{center}
\begin{tabular}{>{\color{accent}\bfseries}l >{\color{white}}c >{\color{highlight}}c}
\toprule
\textcolor{primary}{\textbf{Tipo}} & \textcolor{primary}{\textbf{$f(x)$}} & \textcolor{primary}{\textbf{$f'(x)$}} \\
\midrule
Constante & $c$ & $0$ \\
Identidad & $x$ & $1$ \\
Potencia & $x^n$ & $nx^{n-1}$ \\
Exponencial natural & $e^x$ & $e^x$ \\
Exponencial general & $a^x$ & $a^x \ln a$ \\
Logaritmo natural & $\ln x$ & $\dfrac{1}{x}$ \\
Logaritmo general & $\log_a x$ & $\dfrac{1}{x \ln a}$ \\
\bottomrule
\end{tabular}
\end{center}

\thinline

\subsection{Derivadas Trigonométricas}

\begin{formulabox}
\begin{multicols}{2}
\begin{itemize}[leftmargin=*, label=\textcolor{primary}{$\bullet$}]
\item $\dv{}{x}[\sin x] = \cos x$
\item $\dv{}{x}[\cos x] = -\sin x$
\item $\dv{}{x}[\tan x] = \sec^2 x$
\item $\dv{}{x}[\cot x] = -\csc^2 x$
\item $\dv{}{x}[\sec x] = \sec x \tan x$
\item $\dv{}{x}[\csc x] = -\csc x \cot x$
\end{itemize}
\end{multicols}
\end{formulabox}

\subsection{Reglas de Derivación}

\begin{formulabox}
\textcolor{accent}{\textbf{Reglas básicas:}}
\begin{align*}
\textcolor{highlight}{\text{Suma:}} \quad & (f + g)' = f' + g' \\
\textcolor{highlight}{\text{Producto:}} \quad & (f \cdot g)' = f' \cdot g + f \cdot g' \\
\textcolor{highlight}{\text{Cociente:}} \quad & \left(\frac{f}{g}\right)' = \frac{f' \cdot g - f \cdot g'}{g^2} \\
\textcolor{highlight}{\text{Cadena:}} \quad & (f \circ g)'(x) = f'(g(x)) \cdot g'(x)
\end{align*}
\end{formulabox}

\separator

% ========================================
% SECCIÓN 3: LISTAS Y PASOS
% ========================================
\section{Procedimientos y Métodos}

\subsection{Pasos para derivar funciones complejas}

\begin{enumerate}[label=\textcolor{accent}{\textbf{Paso \arabic*:}}, leftmargin=*, itemsep=8pt]
\item \highlight{Identificar} la estructura de la función (producto, cociente, composición)
\item \highlight{Determinar} qué reglas de derivación aplicar
\item \highlight{Derivar} las funciones componentes (si es necesario)
\item \highlight{Aplicar} la regla correspondiente
\item \highlight{Simplificar} el resultado algebraicamente
\end{enumerate}

\subsection{Propiedades importantes}

\begin{itemize}[label=\textcolor{primary}{$\blacktriangleright$}, leftmargin=*, itemsep=5pt]
\item \highlight{Linealidad:} $(af + bg)' = af' + bg'$ para $a, b \in \R$
\item \highlight{Derivabilidad implica continuidad:} Si $f$ es derivable en $a$, entonces $f$ es continua en $a$
\item \highlight{Puntos críticos:} Los extremos locales ocurren donde $f'(x) = 0$ o $f'$ no existe
\item \highlight{Teorema del Valor Medio:} Conecta la derivada con la pendiente promedio
\end{itemize}

\separator

% ========================================
% SECCIÓN 4: GRÁFICAS CON TIKZ
% ========================================
\section{Visualizaciones con TikZ/PGFplots}

\subsection{Función y su derivada}

\begin{center}
\begin{tikzpicture}
\begin{axis}[
	width=13cm,
	height=9cm,
	axis lines=middle,
	xlabel={$x$},
	ylabel={$y$},
	xlabel style={color=accent, font=\large},
	ylabel style={color=accent, font=\large},
	xmin=-3.5, xmax=3.5,
	ymin=-2, ymax=10,
	samples=150,
	grid=major,
	grid style={color=dark!50, dashed},
	tick style={color=accent},
	legend style={
		at={(0.02,0.98)},
		anchor=north west,
		fill=dark,
		draw=primary,
		text=white,
		font=\small
	},
	title={\textcolor{accent}{Función Cuadrática y su Derivada}},
	title style={font=\large\bfseries}
]

% Función original
\addplot[primary, ultra thick, domain=-3:3, smooth] {x^2};
\addlegendentry{$f(x) = x^2$}

% Derivada
\addplot[accent, ultra thick, domain=-3:3, dashed, smooth] {2*x};
\addlegendentry{$f'(x) = 2x$}

% Punto donde la derivada es cero
\addplot[success, only marks, mark=*, mark size=3pt] coordinates {(0,0)};
\node[color=success, above right] at (axis cs:0,0) {$f'(0)=0$};

\end{axis}
\end{tikzpicture}
\end{center}

\separator

% ========================================
% SECCIÓN 5: ECUACIONES Y DEMOSTRACIONES
% ========================================
\section{Ecuaciones Matemáticas}

\subsection{Ecuaciones numeradas}

Consideremos la integral definida del Teorema Fundamental del Cálculo:

\begin{equation}
\inte{a}{b} f'(x)\,dx = f(b) - f(a)
\label{eq:tfc}
\end{equation}

Podemos referenciar la ecuación~\ref{eq:tfc} más adelante.

\subsection{Sistema de ecuaciones}

Las derivadas de las funciones trigonométricas básicas son:

\begin{align}
\dv{}{x}[\sin x] &= \cos x \\
\dv{}{x}[\cos x] &= -\sin x \\
\dv{}{x}[e^x] &= e^x
\end{align}

\subsection{Fórmulas destacadas}

\mainformula{\pdv{^2 f}{x \partial y} = \pdv{}{x}\left(\pdv{f}{y}\right)}

\separator

% ========================================
% SECCIÓN 6: TEOREMAS Y DEMOSTRACIONES
% ========================================
\section{Teoremas y Demostraciones}

\begin{thm}[Teorema del Valor Medio de Lagrange]
\label{thm:tvm}
Sea $f:[a,b] \to \R$ una función continua en $[a,b]$ y derivable en $(a,b)$. Entonces existe al menos un punto $c \in (a,b)$ tal que:
\[
f'(c) = \frac{f(b)-f(a)}{b-a}
\]
\end{thm}

\begin{proof}
\textcolor{highlight}{\textbf{Demostración:}} 

Definimos la función auxiliar:
\[
g(x) = f(x) - f(a) - \frac{f(b)-f(a)}{b-a}(x-a)
\]

Observamos que $g$ representa la diferencia entre $f$ y la recta secante que une $(a, f(a))$ con $(b, f(b))$.

\textit{Propiedades de $g$:}
\begin{itemize}
\item $g(a) = f(a) - f(a) - 0 = 0$
\item $g(b) = f(b) - f(a) - (f(b) - f(a)) = 0$
\item $g$ es continua en $[a,b]$ y derivable en $(a,b)$
\end{itemize}

Por el \textcolor{accent}{Teorema de Rolle}, existe $c \in (a,b)$ tal que $g'(c) = 0$.

Calculando $g'(c)$:
\[
g'(c) = f'(c) - \frac{f(b)-f(a)}{b-a} = 0
\]

Por lo tanto:
\[
f'(c) = \frac{f(b)-f(a)}{b-a} \qquad \qed
\]
\end{proof}

\begin{cor}
Si $f'(x) = 0$ para todo $x \in (a,b)$, entonces $f$ es constante en $[a,b]$.
\end{cor}

\separator

% ========================================
% SECCIÓN 7: COMANDOS PERSONALIZADOS
% ========================================
\section{Uso de Comandos Personalizados}

Este template incluye múltiples comandos para facilitar la escritura:

\subsection{Conjuntos numéricos}
Los conjuntos $\N$, $\Z$, $\Q$, $\R$ y $\C$ se escriben con comandos simples: \texttt{\textbackslash N}, \texttt{\textbackslash Z}, \texttt{\textbackslash Q}, \texttt{\textbackslash R}, \texttt{\textbackslash C}

\subsection{Derivadas e integrales}
\begin{itemize}
\item Derivada ordinaria: $\dv{y}{x}$ con \texttt{\textbackslash dv\{y\}\{x\}}
\item Derivada de orden $n$: $\dvn{y}{x}{3}$ con \texttt{\textbackslash dvn\{y\}\{x\}\{3\}}
\item Derivada parcial: $\pdv{f}{x}$ con \texttt{\textbackslash pdv\{f\}\{x\}}
\item Integral definida: $\inte{0}{1} f(x)\,dx$ con \texttt{\textbackslash inte\{0\}\{1\}}
\end{itemize}

\subsection{Normas y valores absolutos}
\begin{itemize}
\item Valor absoluto: $\abs{x}$ con \texttt{\textbackslash abs\{x\}}
\item Norma: $\norm{\vect{v}}$ con \texttt{\textbackslash norm\{\textbackslash vect\{v\}\}}
\item Vector: $\vect{u}$ con \texttt{\textbackslash vect\{u\}}
\end{itemize}

\separator

% ========================================
% PIE DE PÁGINA FINAL
% ========================================
\vfill
\begin{center}
\textcolor{primary}{\rule{12cm}{2pt}}\\[10pt]
\textcolor{highlight}{\LARGE\textbf{Fin del Template}}\\[6pt]
\textcolor{accent}{\Large Template Matemáticas - Estilo Institucional Profesional}\\[4pt]
\textcolor{white}{Elaborado en \LaTeX\ | \today}\\[2pt]
\textcolor{highlight}{\small Versión 2.0 - Optimizada y mejorada}
\end{center}

\end{document}
